\subsection{Entwicklungsumgebung für Hardware}
\SecAuth{\nameJS / \nameSH}
\subsubsection{Solidworks\autocite{SOLIDWORKS}}\label{sec:solidworks}
\begin{figwindow}[0,r,\includegraphics[scale=1]{image/logosolidworks.jpg},{Logo SolidWorks\autocite{Solidworks-Logo}}]
SOLIDWORKS ist eine professionelle 3D-CAD-Software (Computer-Aided Design), die es Ingenieuren und Designern ermöglicht, hochpräzise Modelle und Zeichnungen von Produkten zu erstellen. Die Software bietet eine breite Palette von Funktionen, darunter Modellierung, Baugruppenkonstruktion, Zeichnungserstellung, Simulation, Datenmanagement und vieles mehr.
Mit SOLIDWORKS können Anwender komplizierte Bauteile und Baugruppen modellieren, um realistische virtuelle Darstellungen von Produkten zu erstellen. Die Software bietet leistungsstarke Werkzeuge für die Erstellung von Skizzen, die Extrusion von Formen, das Hinzufügen von Details wie Bohrungen und Gewinden sowie die Erstellung komplexer Oberflächen.
Darüber hinaus ermöglicht SOLIDWORKS die Erstellung von Baugruppen, in denen mehrere Teile miteinander verbunden sind. Benutzer können Teile zu Baugruppen zusammenfügen, Bewegungen simulieren und Kollisionen erkennen, um sicherzustellen, dass das Endprodukt korrekt funktioniert.
\end{figwindow}
\vspace{3mm}
%\newpage
\subsubsection{Eagle 7.7.0}\label{sec:Eagle}
\begin{figwindow}[0,r,\includegraphics[scale=3]{image/eagle.png},{Logo Eagle}]
EAGLE\autocite{Eagle} ist die Abkürzung für Einfach Anzuwendender Grafischer Layout-Editor. Die Software besteht aus mehreren Komponenten, Layout-Editor, Schaltplan-Editor, Autorouter und Bauteil-Bibliothek. Mit diesen Komponenten können Schaltpläne gezeichnet werden und danach das Board erstellt werden. Eagle besitzt eine große Bibliothek an Bauteilen. Es gibt jedoch viele Bauteile, die vom Internet heruntergeladen werden können und zur Bibliothek hinzugefügt werden können. Bauteile können auch selbst gezeichnet werden.
\end{figwindow}