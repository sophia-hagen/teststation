\subsubsection{Visual Studio Code\autocite{VScode}}\label{sec:VSC}
\begin{figwindow}[2,r,\includegraphics[scale=0.1]{image/logovscode.png},{Logo VS-Code\autocite{Logo_VSCode}}]
Visual Studio Code, kurz auch VS Code, ist ein Code-Editor. Dieser Editor ist plattformübergreifend für die Betriebssysteme Windows, macOS und Linux verfügbar. VS-Code wird hauptsächlich von einem Team in der Schweiz entwickelt und wird als offenes Projekt auf GitHub verfügbar. Jeden Monat erscheint eine neue Version und es ist das am stärksten unterstütze Projekt auf GitHub. VS-Code ist geeignet für die Programmierung in vielen verschiedenen Programmiersprachen. Beispiele für Programmiersprachen sind, Batch, C++, Python und viele mehr. Durch Plug-ins kann VS-Code erweitert werden. 
\end{figwindow}
\newpage
\subsubsection{Plotly Dash}\label{sec: dash}
Plotly Dash\autocite{Dash} ist ein Open-Source-Framework für interaktive Webanwendungen in Python. Es erleichtert Entwicklern das Erstellen von Webanwendungen, ohne dass Kenntnisse in HTML oder CSS notwendig sind. Die Apps sind komplett in Python geschrieben. Plotly.js wird für Datenvisualisierung genutzt, was viele interaktive Diagramme ermöglicht. Eine Dash-App besteht aus zwei Teilen: \\
\vspace{3mm}
Das Layout bestimmt das Aussehen der App und ordnet die visuellen Elemente an. Es wird mit vordefinierten Dash-Komponenten erstellt, einschließlich HTML-Elementen und Plotly-Diagrammen. \\
\vspace{5mm}
Callbacks machen die App interaktiv. Sie werden aktiviert, wenn Benutzer mit der App interagieren, z.B. einen Button klicken oder eine Auswahl aus einem Dropdown-Menü treffen. Diese Funktionen nehmen Benutzereingaben entgegen, führen Berechnungen oder Datenabfragen durch und aktualisieren die Anzeigeelemente mit den neuen Daten.

\subsubsection{TeraTerm}\label{sec:teraterm}
TeraTerm\autocite{TeraTerm} ist ein Open-Source-Emulationstool für Microsoft Windows. Es ist ein Terminal-Emulator, der die Kommunikation mit entfernten Systemen über verschiedene Protokolle ermöglicht - Telnet, SSH (Secure Shell) und serielle Verbindungen. TeraTerm unterstützt auch Makro-gesteuerte Logik, was es für automatisierte Aufgaben und Tests nützlich macht.
