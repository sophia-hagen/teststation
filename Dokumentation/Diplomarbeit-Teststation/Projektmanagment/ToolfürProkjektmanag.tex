\subsection{Projektmanagementtool}
Für das Projektmanagement wurden zwei Programme verwendet. Diese Programme wurden genutzt, um die Arbeit im Blick zu behalten und die Unterlagen für alle zugänglich zu machen. 

\subsubsection{GanttProject}
GanttProject\autocite{GanttProject} wurde verwendet, um einen Projektplan zu erstellen. Es wurden verschiedene Arbeitspakete definiert, die zu einem Bestimmten Datum abgeschlossen werden müssen. In GanttProject können verschiedene Teammitglieder den Arbeitspaketen zugewiesen werden. So entsteht das Ressourcendiagramm. Im Ressourcendiagramm wird die Auslastung einer Person oder einer Ressource über eine Zeitspanne dargestellt. Die erstellte File ist in der GitHub Repository.

\subsubsection{GitHub}
Auf GitHub wurde eine Repository für die Diplomarbeit erstellt. So hat jedes Teammitglied Zugriff auf die verschiedenen Files. In der Repository sind vier Ordner zu finden. 
\begin{itemize}
    \item firmware - Testprogrammfile
    \item api - API-File
    \item Projektmanagment - Gantt-Projekt
    \item Dokumentation 
\end{itemize}
\pagebreak
Die Repository kann erreicht werden, indem der unten angeführte QR-Code gescannt wird. 
\vspace{5mm}
\begin{figure}[H]
    \centering
    \includegraphics[scale=0.8]{image/QRCode.png}
    \caption{QR-Code GitHub Repository}
    \label{fig:enter-label}
\end{figure}