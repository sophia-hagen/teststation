\subsection{Magnetfeld}\label{sec:Magnetfeld}
Der CubeSat befindet sich später in einer Umlaufbahn der Magnetosphäre, deshalb soll der CubeSat in der Testkammer einem Magnetfeld ausgesetzt werden.\\
\vspace{5mm}
Die Magnetosphäre ist eine Region die die Erde umschließt, um die Oberfläche von Sonnenwinden zu schützen. Diese Region besitzt ein Magnetfeld. Das Magnetfeld der Magnetosphäre wird durch den Erdkern erzeugt. Die Grenze für die Magnetosphäre ist unbekannt, da diese Region weit ins Weltall reicht.\\
\vspace{5mm}
Es wurden Neodym Dauermagnete\autocite{Magnet} eingesetzt, um das Magnetfeld zu erzeugen.\\
\vspace{3mm}
Ein Neodym Magnete erzeugt ein Magnetfeld im Bereich von einigen Hundert bis mehreren Tausend Gauss. Das Magnetfeld der Erde ist sehr schwach. Es besitzt eine größe von 0,7 Gauss.\\
\vspace{3mm}
Diese Magnete werden in der Kammer befestigt. Die ursprüngliche Idee war, die Magnete am Gyroskop zu befestigen. Da diese Magneten jedoch ein sehr starkes Magnetfeld erzeugen, werden sie weiter weg platzier.\\
\vspace{5mm}
\begin{figure}[H]
    \centering
    \includegraphics[scale=0.4]{image/Magnet.png}
    \caption{Dauermagnet\autocite{Beispielbild_Magnet}}
    \label{fig:enter-label}
\end{figure}

