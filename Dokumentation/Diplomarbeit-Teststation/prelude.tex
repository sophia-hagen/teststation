\documentclass[hidelinks,12pt,a4paper,twoside]{article}

\usepackage[utf8]{inputenc}
% Paket für die Anpassung von Layout und Abständen
\usepackage{setspace} 	
\usepackage{array} 
\usepackage{tocloft}

\usepackage[bottom,hang,stable]{footmisc}
\usepackage[font=small]{caption}
\usepackage{subcaption}


\setlength{\cftsubsecnumwidth}{3em}

\makeatletter                               
\renewcommand*\l@table{\@dottedtocline{1}{3em}{5em}}
\renewcommand*\l@figure{\@dottedtocline{1}{1.5em}{4em}}
\makeatother

\usepackage{longtable}

\usepackage{graphicx} % Required for inserting images
\usepackage{fancyhdr}

\usepackage{amssymb}
% Überschriftenabstand anpassen
\usepackage{titlesec}
\usepackage[ngerman]{babel}
\usepackage[document]{ragged2e}
\usepackage[top=3.5cm,left=3cm,right=2cm,bottom=2.5cm,headsep=0.3in,headheight=1in]{geometry}

\usepackage{charter}
\usepackage{adjustbox}
\usepackage{csquotes}
\usepackage[citestyle=authortitle, backend=bibtex]{biblatex}
\usepackage{minted}
\usepackage{picinpar}
\usepackage{wrapfig}
\usepackage{pdfpages} %pdf 
\usepackage{varwidth}
\usepackage{hyperref} % Paket für Hyperlinks
					
					
				
\usepackage{amsmath} 	

\usepackage{listings,xcolor}
\usepackage{listingsutf8} 
\usepackage{float} 
\usepackage{footnote}
\usepackage{enumitem}
\usepackage{longtable}

\usepackage{bold-extra}


% allgemein gültige Formateinstellungen
% -------------------------------------
\renewcommand{\familydefault}{\sfdefault} 	% set Font-Family to a similar font like Arial
\raggedbottom 								% Abschnitte werden nicht auseinander gezogen, um den restlichen Platz auf einer Seite zu füllen
%\raggedright 								% gesamtes Dokument linksbündig (auch Paragraphen) OHNE Silbentrennung

\usepackage{ragged2e}						% gesamtes Dokument linksbündig (auch Paragraphen) MIT Silbentrennung
\RaggedRight

\onehalfspacing 							% 1.5-facher Abstand zwischen den Zeilen

\renewcommand{\arraystretch}{1.05} 			% Vergrößerung des Abstands zwischen den Tabellenzeilen
\newcolumntype{C}[1]{>{\centering\arraybackslash}p{#1}} % erstellen eines Spaltentyps mit zentriertem Inhalt unter der Angabe einer Spaltenbreite
\newcolumntype{L}[1]{>{\raggedright\arraybackslash}p{#1}} % erstellen eines Spaltentyps mit linksbündigem Inhalt unter der Angabe einer Spaltenbreite

\setlength{\footnotemargin}{0.5cm} 			% Abstand zwischen Fußnotennummer und -text
\setlist{nosep} 							% zusätzliche Abstände bei Aufzählungen entfernen
\setlength{\parskip}{1em} 					% Abstand nach einem Absatz

\skip\footins 30pt

% Abstände vor und nach Überschriften auf den verschiedenen Ebenen
\titlespacing\section       {0pt}{0pt plus 0pt minus 1pt}{0pt plus 1pt minus 1pt}
\titlespacing\subsection    {0pt}{0pt plus 0pt minus 1pt}{0pt plus 1pt minus 1pt}
\titlespacing\subsubsection {0pt}{0pt plus 0pt minus 1pt}{0pt plus 1pt minus 1pt}