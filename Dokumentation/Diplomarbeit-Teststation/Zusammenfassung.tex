
\section{Zusammenfassung}
{Das Spaceteam der TU-Wien benötigt eine Teststation, um die Zuverlässigkeit des CubeSat zu überprüfen. Dafür soll eine Kammer gebaut werden, in der verschiedene Tests mit dem Satelliten durchgeführt werden können. Die Kammer soll bei einer Temperatur von 0°C bis 30°C betrieben werden können. In der Kammer befinden sich verschiedene Sensoren, um einen Vergleich zwischen den Sensoren auf dem CubeSat und in der Teststation herzustellen. Durch diesen Ist-Soll Vergleich können Fehler auf dem CubeSat erkannt werden und fehlerhafte Bauteile ausgetauscht werden. Um diesen Ist-Soll-Vergleich zu veranschaulichen und um alle Komponenten anzusteuern, soll eine benutzerfreundliche Webapplikation entwickelt werden. Mithilfe eines Gyroskops wird der CubeSat in x-, y- und z-Richtung rotiert. Mit den Sensoren auf dem Satelitt können die Auswirkungen dieser Rotation erfasst werden und mithilfe der Webapplikation dargestellt werden.}