\section{Dokumentationsaufteilung}

Unten wird die Aufteilung der Dokumentation veranschaulicht. Im Kapitel \ref{sec: einteilung} Arbeitsaufteilung ist zu erkennen welches Teammitglied für was verantwortlich war. Dementsprechend musste jedes Teammitglied diesen Teil dokumentieren. 

\subsection{\nameSH}
\begin{itemize}
    \item Kapitel \ref{sec:pflichtenheft} Pflichtenheft
    \item Hardware
    \begin{itemize}
        \item Kapitel \ref{sec:Eagle} Eagle 7.7.0 
        \item Kapitel \ref{sec:Sensoren in der Testkammer} Sensoren in der Teststation 
        \item Kapitel \ref{sec:UV-Lampe} UV-Lampe 
        \item Kapitel \ref{sec:Magnetfeld} Magnetfeld 
    \end{itemize}
    \item Software
    \begin{itemize}
        \item Kapitel \ref{sec:VSC} Visual Studio Code 
        \item Kapitel \ref{sec:Tasterabfrage} Tasterabfrage 
        \item Testprogramm
        \begin{itemize}
            \item Kapitel \ref{sec:Testprogramm Temp} Temperatursensor  
            \item Kapitel \ref{sec:Testprogramm UV-Sensor} UV-Sensor 
            \item Kapitel \ref{sec:Testprogramm Drucksensor} Drucksensor 
            \item Kapitel \ref{sec:Testprogramm UV-Lampe} UV-Lampe 
        \end{itemize}
    \end{itemize}
    \item Steuerung
    \begin{itemize}
        \item Kapitel \ref{sec:Grafische Oberfläche} Grafische Oberfläche Raspberry Pi 
        \item Kapitel \ref{sec:Remote Zugriff} Remote Zugriff 
        \item Kapitel \ref{sec:API} Fast API 
    \end{itemize}
    \item Kapitel \ref{sec:Projektmanagement} Projektmanagement
\end{itemize}

\subsection{\nameJS}
\begin{itemize}
    \item Hardware
    \begin{itemize}
        \item Kapitel \ref{sec:solidworks} Solidworks 
        \item Kapitel \ref{sec:Gehäuse} Gehäuse 
        \item Kapitel \ref{sec:gyroskop} Gyroskop 
        \item Kapitel \ref{sec:relai} Relais
        \item Kapitel \ref{sec:Pneumatik} Pneumatik
        \item Kapitel \ref{sec:LED} LED-Streifen 
        \item Kapitel \ref{sec:sicherheit} Sicherheit 
    \end{itemize}
    \item Software
    \begin{itemize}
        \item Kapitel \ref{sec:thonny} Thonny 
        \item Testprogramm
        \begin{itemize}
            \item Kapitel \ref{sec:GUI} GUI 
            \item Kapitel \ref{sec:Button} Button 
            \item Kapitel \ref{sec:test motor} Motor
            \item Kapitel \ref{sec:test notaus} Notaus 
            \item Kapitel \ref{sec:test Led} LED-Streifen
            \item Kapitel \ref{sec:test end} Endschalter
        \end{itemize}
    \end{itemize}
    \item Steuerung
    \begin{itemize}
        \item Kapitel \ref{sec:raspi} \raspi 
    \end{itemize}
\end{itemize}

\subsection{\nameCZ}
\begin{itemize}
	\item Hardware
	\begin{itemize}
		\item Kapitel \ref{sec:sensoraufcube} Sensoren auf dem CubeSate 
	\end{itemize}
	\item Software
	\begin{itemize}
		\item Kapitel \ref{sec: dash} Plotly Dash
		\item Kapitel \ref{sec:teraterm} TeraTerm
		
		\item Kapitel \ref{Verbindung} Verbindungsmöglichkeit 
		\item Kapitel \ref{über} Übertragungsmethode 
		\item Kapitel \ref{empfang} Empfangsroutine
		\item Kapitel \ref{bib} Bibliothek „STS1\_Sensors.py “ 
		\item Kapitel \ref{aus} Auslesen der Sensorik mittels „STS1\_Sensors.py“
		
	\end{itemize}
	\item Steuerung
	\begin{itemize}
		\item Kapitel \ref{Web} Webapplikation 
	\end{itemize}
	\item Testen
	\begin{itemize}
		\item Kapitel \ref{Messungen} Messungen
	\end{itemize}
\end{itemize}
\newpage
\subsection{\nameSB}
\begin{itemize}
    \item Hardware
    \begin{itemize}
        \item Kapitel \ref{sec: CubeSat} CubeSat 
        \item Kapitel \ref{sec: Halterung} Halterung 
        \item Kapitel \ref{sec: Lüftung} Lüfter 
        \item Kapitel \ref{sec: Kühlung} Kühlung
    \end{itemize}
    \item Software
    \begin{itemize}
        \item Testprogramm
        \begin{itemize}
            \item Kapitel \ref{sec: kühlung test} Kühlung 
            \item Kapitel \ref{sec: lüfter test} Lüfter 
        \end{itemize}
    \end{itemize}
    \item Testen
    \begin{itemize}
    	\item Kapitel \ref{sec_routine} Testroutine
    \end{itemize}
\end{itemize}
