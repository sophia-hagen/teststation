\subsubsection{Tür}
Es soll möglich sein, die Kammer zu verschließen, um äußerliche Einwirkungen zu verhindern. Zusätzlich sollte die Kammer luftdicht sein, denn weder warme oder kalte Luft sollte entfliehen können. Da auch das Gyroskop akustisch etwas laut ist, dient die Tür auch als Geräusch Verminderung. 
Der Rahmen für die Tür wird aus Holzleisten gefertigt, die eine Stärke von 2cm und eine breite von 4cm haben.\\
\vspace{3mm}
\begin{table}[H]
    \centering
    \begin{tabular}{ | c | c | } 
  \hline
   \textbf{Bezeichnung} & \textbf{Stückzahl}\\ 
  \hline
   Holzleiste 4x2x70cm & 2\\ 
  \hline
   Holzleiste 4x2x56cm & 2 \\ 
  \hline
  Holzleiste 0.2x54x68cm & 1\\
  \hline
  Edelstahlgriff Lochabstand 128mm & 1\\
  \hline
  M3 Schraube mm & 2\\
  \hline
  M3 Mutter & 2\\
  \hline
  Silikon Kleber & 1\\
  \hline
\end{tabular}
    \caption{Stückliste PVC-Platte}
\end{table}
\newpage
Die Holzleisten werden auf Gehrung verbunden, denn durch diese Methode vergrößert sich die Kontaktfläche und die Stabilität wird erhöht. 
Plexiglas ist ein transparentes Kunststoffmaterial, das als Ersatz für Glas verwendet wird. Es ist leichter, günstiger und bruchsicherer als Glas. Das Plexiglas hat eine Stärke von 2mm und wird in eine Nut im Rahmen eingelassen, um eine einfache Befestigung zu gewährleisten.  \\
\vspace{3mm}
Zusammenbau:
Um sicherzustellen, dass das Plexiglas während des Zusammenbauens und Verleimens ordnungsgemäß in den Rahmen eingesetzt wurde, waren folgende Schritte erforderlich:\\
\vspace{3mm}
1. Vorbereutung der Leisten
\begin{itemize}
    \item ˆ Zunächst wurde eine lange Leiste mit der Nut nach unten auf die Arbeitsfläche (Fläche X) gelegt.
    \item Eine kurze Leiste mit der Nut nach unten wurde bündig an der rechten Seite der langen Leiste platziert. Dieser Schritt wurde auch auf der linken Seite der langen Leiste mit einer weiteren kurzen Leiste wiederholt.
\end{itemize}
2. Fixierung mit Klebeband:
\begin{itemize}
    \item Die drei Leisten wurden mit Klebeband zusammengehalten, um zu verhindern, dass sie beim Verleimen verrutschen.
\end{itemize}
\vspace{3mm}
3. Vorbereitung für den Leimauftrag:
\begin{itemize}
    \item Die drei Leisten wurden zusammen um 180° gedreht. Holzleim wurde großzügig auf die Oberfläche aufgetragen und gleichmäßig verteilt.
\end{itemize}

4. Einsetzen des Plexiglases:
\begin{itemize}
    \item Das Plexiglas wurde vorsichtig in die Nut der langen Leiste eingelegt. Hierbei konnte eine zweite Person von Vorteil sein, um das Plexiglas zu halten, während die anderen beiden Leisten um 90° angehoben wurden. Dabei wurde darauf geachtet, dass das Plexiglas auch in den kurzen Leisten eingespannt war.
\end{itemize}

5. Abschluss des Rahmens:
\begin{itemize}
    \item Abschließend wurde die verbleibende zweite lange Leiste auf die kurzen Seiten und das Plexiglas gelegt.
\end{itemize}
\newpage
6. Fixierung und Trocknung:
\begin{itemize}
    \item Schraubzwingen wurden verwendet, um den Rahmen fest zusammenzuspannen und den Leim trocknen zu lassen.
Durch diese Schritte wurde sichergestellt, dass das Plexiglas ordnungsgemäß in den Rahmen eingesetzt wurde und während des Verleimens nicht verrutschte.Mit einem Silikonkleber wird das Plexiglas mit der Nut verklebt und verdichtet. Nachdem das Silikon ausgetrocknet war, war die Tür dicht.
Als Türgriff wird ein Edelstahlgriff mit einem Lochabstand von 13cm verwendet. Der Griff kann dank einer soliden Verschraubung sicher und stabil angebracht werden, was eine zuverlässige Nutzung gewährleistet.
Die Tür wird mit einfachen Scharnieren an dem Gestell befestigt.
\end{itemize}

